\section{Related work}\label{sec:relatedwork}
Configuration management tools are a frequently researched topic as their presence may lead to improvements in overall workflow. As these tools are mainly aimed at corporate infrastructures, the available literature commonly focuses on specific use cases from an Enterprise standpoint. Frequently these reports present use cases related to security aspects \cite{dotson2014security} or the repeatability of deployments \cite{ruiz2015reconstructable}. Additionally, as time progresses and new tools become available, product comparisons are drawn up. Hardion et al. \cite{Hardion2013} present a high level comparison of various configuration management tools on the market from the standpoint of a big research facility. Regarding the actual configuration of such tools, Collard et al. \cite{Collard2015} describe a method for verifying the configuration of Puppet. However, this paper exclusively focuses on environments with a single configuration management system. 

This research project takes a real world migration use case as a starting point. Little official research related to this specific topic is available. Dawson et al. \cite{dawson_hall_hecht_2014} present a motivation for such a migration trajectory and identify Puppet as the most influential configuration management tool. Ansible is identified as a common migration target. In a blog post, Zunker \cite{zunker_2014} goes into more depth and presents a starting point for performing a gradual migration between Puppet and Ansible. He describes a conceptual method for piping output from Puppet to Ansible and talks about interpreting the output from each tool. This project will form an extension of the work of Zunker.   

Moreover, as of late, new concepts for configuration management tools have been surfacing. \texttt{mgmt} by Shubin \cite{shubin2016} is a self proclaimed 'next generation' configuration management prototype.  Up until now, most configuration management tools have been employing a push or pull-based client-server model. \texttt{mgmt} uses a distributed architecture which allows for parallelization and a distributed topology. Due to the way \texttt{mgmt} is built, the tool may be suitable for a gradual migration trajectory. 
