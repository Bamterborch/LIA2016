\section{Conclusion}\label{sec:conclusion}
Migration between different configuration management systems is a valid use case for large installations. Puppet origins from 2005 and Ansible origins from 2012. Puppet and Ansible are the most used CMS's and have totally different architectures and use cases. Companies should choose the best system that will fulfill there specific needs. As the table \ref{table:comparisson} displays the three systems have there own use cases. Developers of these systems do not pay attention to removing the client software from the managed host. But as the models in section \ref{sec:methodology} show it is feasible to migrate between systems in a way that will not result in service downtime in a large scaled environment. The method doesn't  

\subsection{Future Work}\label{subsec:futurework}
This report primarily focuses on managing generic virtual machine instances with Puppet and Ansible and does not take Docker containerization in consideration. Deploying Docker containers on a large scale is currently primarily being done via Dockerfiles, which contain a sequence of pre-defined commands in order to deploy a service with a given configuration. Problems occur when the services to be deployed become overly complex and require a large set of commands. At that point they require relatively static and unmanageable Dockerfiles. Configuration management tools like the ones discussed in this report are currently looking into supporting fine-grained configuration management features for various types of container platforms. However, as such features become available the line between 'true' orchestration tools like Kubernetes \cite{kubernetes_2016} become less obvious. It would be interesting to see how orchestration tools at this point in time relate to configuration management tools and whether a clear distinction can be made in the near future.

Although we don't focus on \texttt{mgmt} in detail in this report, it is undeniably an interesting development. As this new tool is still in an experimental phase, it does not have a definitive Domain Specific Language in place yet. This means that a migration strategy to a distributed model is currently out of scope. However, when the development of \texttt{mgmt} takes on a more solid form, it would be interesting to see how a migration from Puppet to \texttt{mgmt} would play out. James \cite{frank_2016} provides a starting point for transpiling Puppet modules to \texttt{mgmt} resource graphs by using a series of complex regular expressions. However, further investigation on this topic is required.
