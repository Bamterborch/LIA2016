\section{Conclusion}\label{sec:conclusion}
Where corporate merges take place, the IT department is one of the first departments that need to merge. These merges come with great challenges.One of those questions will be: What tools to use to manage the merged infrastructure? This reports looks into configuration management systems for server management in case of a corporate merge. What prerequisites are needed to do an automated merge. A virtual lab environment was setup to see if the model worked as predicted. A lot of technical constraints could be easily overcome which makes the model a good starting point for future merges and researches into migration between CMS's. In this paper we present a method for performing an in place migration between a pull and a push based configuration management system. More specifically, we have examined the implications of migrating a node from being managed by Puppet to Ansible. Additionally we have assessed the possibilities of migrating to the relatively new and distributed configuration management system \texttt{mgmt}. The \texttt{mgmt} software is not ready to be used in a large scale environment. But we are sure that with the right amount of effort the approach is the next step to take for configuration management systems. Depending on one location for configuration management brings in some risks. 

We conclude on the notion that an in place migration on production machines is feasible without any perceivable effect for end users. However, the migration between systems is heavily dependent on the architectural model of the tool. Therefore, extrapolating the model as presented in this report to a generic model would be ill advised. Next to this some other research is done to automate the baseline for yet installed services on a machine. Which makes the initial phase of the migration easier.  


\subsection{Future Work}\label{subsec:futurework}
This report primarily focuses on managing generic virtual machine instances with Puppet and Ansible and does not take Docker containerization in consideration. Deploying Docker containers on a large scale is currently primarily being done via Dockerfiles, which contain a sequence of pre-defined commands in order to deploy a service with a given configuration. Problems occur when the services to be deployed become overly complex and require a large set of commands. At that point they require relatively static and unmanageable Dockerfiles. Configuration management tools like the ones discussed in this report are currently looking into supporting fine-grained configuration management features for various types of container platforms. However, as such features become available the line between 'true' orchestration tools like Kubernetes \cite{kubernetes_2016} become less obvious. It would be interesting to see how orchestration tools at this point in time relate to configuration management tools and whether a clear distinction can be made in the near future.

Although we don't focus on \texttt{mgmt} in detail in this report, it is undeniably an interesting development. As this new tool is still in an experimental phase, it does not have a definitive Domain Specific Language in place yet. This means that a migration strategy to a distributed model is currently out of scope. However, when the development of \texttt{mgmt} takes on a more solid form, it would be interesting to see how a migration from Puppet to \texttt{mgmt} would play out. James \cite{frank_2016} provides a starting point for transpiling Puppet modules to \texttt{mgmt} resource graphs by using a series of complex regular expressions. However, further investigation on this topic is required.

This research looked into a small part of the migration between Puppet and Ansible. The installation of the Apache service is well documented and therefore we were able to migrate without any downtime. But other applications behave different or cannot be managed by one of the management systems. Research can be done for the 100 most used applications to see if this model is still feasible on these applications. 
