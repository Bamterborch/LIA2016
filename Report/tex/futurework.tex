\section{Future work}\label{sec:futurework}
This report primarily focuses on managing generic virtual machine instances with Puppet and Ansible and does not take Docker containerization in consideration. Deploying Docker containers on a large scale is currently primarily being done via Dockerfiles, which contain a sequence of pre-defined commands in order to deploy a service with a given configuration. Problems occur when the services to be deployed become overly complex and require a large set of commands. At that point they require relatively static and unmanageable Dockerfiles. Configuration management tools like the ones discussed in this report are currently looking into supporting fine-grained configuration management features for various types of container platforms. However, as such features become available the line between 'true' orchestration tools like Kubernetes \cite{kubernetes_2016} become less obvious. It would be interesting to see how orchestration tools at this point in time relate to configuration management tools and whether a clear distinction can be made in the near future.

Although we don't focus on \texttt{mgmt} in detail in this report, it is undeniably an interesting development. As this new tool is still in an experimental phase, it does not have a definitive Domain Specific Language in place yet. This means that a migration strategy to a distributed model currently is out of scope. However, when the development of \texttt{mgmt} takes on a more solid form, it would be interesting to see how a migration from Puppet to \texttt{mgmt} would play out. James \cite{frank_2016} provides a starting point for transpiling Puppet modules to \texttt{mgmt} resource graphs by using a series of complex regular expressions. However, further investigation is required.
