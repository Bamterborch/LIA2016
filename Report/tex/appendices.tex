\begin{appendices}
  \renewcommand\thetable{\thesection\arabic{table}}
  \renewcommand\thefigure{\thesection\arabic{figure}}

  \section{Puppet Apache module} \label{app:puppetmodule}
  \begin{lstlisting}
   # This class definition depends on the puppetlabs/apache module
   # and installs the Apache server on the client. Additionally
   # it checks for the existence of a simple index.php file. 

   class pe_apache_app {

   # class { 'apache':
   #   mpm_module => 'event',
   # }

   include apache::mod::php

   apache::vhost { 'pe_apache_app':
     port     => '80',
     docroot  => '/var/www/',
     priority => '10',
   }

   file { '/var/www/pe_apache_app/index.php':
     ensure  => file,
     content => "This server is managed by Puppet!",
     mode    => '0644',
   }
  }
  \end{lstlisting}

  \newpage
  \section{Puppet agent removal} \label{app:puppetagent}
  \begin{lstlisting}
   # This class definition initiates a deinstallation of the Puppet
   # agent on the client machine. The placement of the uninstallation
   # script is environment dependent.
       
   class pe_removal_app {

     exec { 'download_uninstaller':
        command => wget --no-check-certificate https://puppet-master.puppet.local:8140/packages/current/uninstall.bash;
        wget --no-check-cerificate https://puppet-master.puppet.local:8140/packages/current/utilities,
        path    => '/usr/local/bin/:/bin/',
        cwd     => "/tmp/",
     }     

     # Puppet adheres to the 'before' statement, thus ensuring that  
     # the required files are present prior to executing the uninstaller. 
     # Also refer to results section in this report.
 
     file { '/tmp/uninstall.sh': 
        ensure => present,
        before => Exec['run_uninstaller'] 
     }

     file { '/tmp/uninstall.sh': 
        ensure => present,      
        before => Exec['run_uninstaller']                  
     }

     exec { 'run_uninstaller':
        # Remove database, purge and assume yes
        command => 'bash -c /tmp/puppet-enterprise-uninstaller -d -p y',
        path    => '/usr/local/bin/:/bin/',
        user	=> root,
        cwd     => "/tmp/",
     }
   }
 
  \end{lstlisting}
  
  \newpage  
  \section{Ansible migration playbook} \label{app:ansiblemigration}
  \begin{lstlisting}
  # This playbook ensures the Puppet master is present in the hostfile
  # which is a natural requirement for downloading the Puppet agent 
  # installer script.
  ---
  - hosts: slave02
    sudo: yes
    tasks:
     - lineinfile: dest=/etc/hosts line="172.16.175.131 puppet-master.puppet.local" state=present
     - name: Install puppet agent
       shell: curl -k https://puppet-master.puppet.local:8140/packages/current/install.bash | sudo bash
  \end{lstlisting}

  \newpage
  \section{Ansible Apache playbook} \label{app:ansibleplaybook}
  \begin{lstlisting}
  # This playbook installs an Apache webserver if it isn't present
  # on the client machine yet. Subseqently it removes any existing
  # (default) index.html and index.php files and creates a new
  # managed index.php file.  
  ---
  - hosts: webservers
    sudo: yes
    tasks:
      - name: install apache2
        apt: name=apache2 update_cache=yes state=latest

      - name: enabled mod_rewrite
        apache2_module: name=rewrite state=present
        notify:
          - restart apache2

      - name: delete old apache index.html
        file: path=/var/www/html/index.html state=absent

      - name: create new awesome index.php
        file: path=/var/www/html/index.php state=file

      - lineinfile: dest=/var/www/html/index.php regexp='.' state=absent
      - lineinfile: dest=/var/www/html/index.php line="This machine is managed by Ansible!" state=present
        notify:
          - restart apache2

  handlers:
    - name: restart apache2
      service: name=apache2 state=restarted
  \end{lstlisting}
  
   \newpage
  \section{Mgmt Graph definition} \label{app:mgmtgraph}
  \begin{lstlisting}
# This graph definition performs an 'apt update' and an 'apt-upgrade' prior to
# installing the apache2 package. Subsequently the files to be hosted by the 
# web server are generated. In parallel a series of sleep operations are ran. 
---
graph: apache2
comment: Parallel installation of Apache2 (Proof of Concept)
resources:
  exec:
  - name: update
    cmd: apt update
    shell: ''
    timeout: 0
    watchcmd: ''
    watchshell: ''
    ifcmd: ''
    ifshell: ''
    pollint: 0
    state: present
  - name: upgrade
    cmd: apt upgrade
    shell: ''
    timeout: 0
    watchcmd: ''
    watchshell: ''
    ifcmd: ''
    ifshell: ''
    pollint: 0
    state: present
  - name: sleep1
    cmd: sleep 10s
    shell: ''
    timeout: 0
    watchcmd: ''
    watchshell: ''
    ifcmd: ''
    ifshell: ''
    pollint: 0
    state: present
  - name: sleep2
    cmd: sleep 10s
    shell: ''
    timeout: 0
    watchcmd: ''
    watchshell: ''
    ifcmd: ''
    ifshell: ''
    pollint: 0
    state: present
  - name: sleep3
    cmd: sleep 10s
    shell: ''
    timeout: 0
    watchcmd: ''
    watchshell: ''
    ifcmd: ''
    ifshell: ''
    pollint: 0
    state: present
  file:
  - name: index
    path: "/var/www/html/index.php"
    content: |
      <head>
        <link rel="stylesheet" type="text/css" href="mystyle.css">
      </head>
      <p>This server is managed by MGMT.</p>
    state: exists
  - name: style
    path: "/var/www/html/style.css"
    content: |
      body {
        background-color: lightblue;
      }
    state: exists
  pkg:
  - name: apache2
    state: installed
  svc:
  - name: apache2
    meta:
      autoedge: false
    state: started
edges:
- name: e1
  from:
    res: exec
    name: update
  to:
    res: exec
    name: upgrade
- name: e2
  from:
    res: exec
    name: upgrade
  to:
    res: pkg
    name: apache2
- name: e3
  from:
    res: pkg
    name: apache2
  to:
    res: file
    name: index
- name: e4
  from:
    res: pkg
    name: apache2
  to:
    res: file
    name: style
- name: e5
  from:
    res: exec
    name: sleep1
  to:
    res: exec
    name: sleep2
- name: e6
  from:
    res: exec
    name: sleep1
  to:
    res: exec
    name: sleep3

  \end{lstlisting}

\end{appendices}


