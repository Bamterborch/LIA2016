\begin{appendices}
  \renewcommand\thetable{\thesection\arabic{table}}
  \renewcommand\thefigure{\thesection\arabic{figure}}

  \section{Puppet Apache module} \label{app:puppetmodule}
  \begin{lstlisting}
   # This class definition depends on the puppetlabs/apache module
   class pe_apache_app {

   class { 'apache':
     mpm_module => 'prefork',
   }

   include apache::mod::php

   apache::vhost { 'pe_apache_app':
     port     => '80',
     docroot  => '/var/www/',
     priority => '10',
   }

   file { '/var/www/pe_apache_app/index.php':
     ensure  => file,
     content => "This server is managed by Puppet!",
     mode    => '0644',
   }
  }
  \end{lstlisting}
  \newpage

  \section{Puppet agent removal} \label{app:puppetagent}
  \begin{lstlisting}
   # This class definition initiates a deinstallation of the Puppet
   # agent on the client machine. The placement of the uninstallation
   # script is environment dependent.
       
   class pe_removal_app {

     exec { 'download_uninstaller':
        command => wget --no-check-certificate https://puppet-master.puppet.local:8140/packages/current/uninstall.bash;
        wget --no-check-cerificate https://puppet-master.puppet.local:8140/packages/current/utilities,
        path    => '/usr/local/bin/:/bin/',
        cwd     => "/tmp/",
     }     

     # Puppet runs with a semi-random order. Therefore the required 
     # files are ensured to be present prior to executing the uninstaller. 
     # Also refer to background section in this report.
 
     file { '/tmp/uninstall.sh': 
        ensure => present,
        before => Exec['run_uninstaller'] 
     }

     file { '/tmp/uninstall.sh': 
        ensure => present,      
        before => Exec['run_uninstaller']                  
     }

     exec { 'run_uninstaller':
        # Remove database, purge and assume yes
        command => 'bash -c /tmp/puppet-enterprise-uninstaller -d -p y',
        path    => '/usr/local/bin/:/bin/',
        user	=> root,
        cwd     => "/tmp/",
     }
   }
 
  \end{lstlisting}
  
  \newpage  
  \section{Ansible migration playbook} \label{app:ansiblemigration}
  \begin{lstlisting}
  ---
  - hosts: slave02
    sudo: yes
    tasks:
     - lineinfile: dest=/etc/hosts line="172.16.175.131 puppet-master.puppet.local" state=present
     - name: Install puppet agent
       shell: curl -k https://puppet-master.puppet.local:8140/packages/current/install.bash | sudo bash
  \end{lstlisting}

  \newpage
  \section{Ansible Apache playbook} \label{app:ansibleplaybook}
  \begin{lstlisting}
    ---
    - hosts: webservers
      sudo: yes
      tasks:
        - name: install apache2
          apt: name=apache2 update_cache=yes state=latest

        - name: enabled mod_rewrite
          apache2_module: name=rewrite state=present
          notify:
            - restart apache2

        - name: delete old apache index.html
          file: path=/var/www/html/index.html state=absent

        - name: create new awesome index.php
          file: path=/var/www/html/index.php state=file

        - lineinfile: dest=/var/www/html/index.php regexp='.' state=absent
        - lineinfile: dest=/var/www/html/index.php line="This machine is managed by Ansible!" state=present
          notify:
            - restart apache2

    handlers:
      - name: restart apache2
        service: name=apache2 state=restarted
  \end{lstlisting}

\end{appendices}


