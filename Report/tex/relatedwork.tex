\section{Related work}\label{sec:relatedwork}
Configuration management systems are a frequently researched topic by Enterprises, as their presence in an IT landscape may lead to significant improvements in overall workflow. As these tools are mainly aimed at managing and maintaining corporate infrastructures, the available literature commonly focuses on specific use cases from an Enterprise standpoint. Frequently these reports present use cases related to security aspects such as infrastructure hardening \cite{dotson2014security}, the repeatability of deployments \cite{ruiz2015reconstructable} or verifying the actual configuration of such tools. Collard et al. \cite{Collard2015} describe a method for verifying the configuration of Puppet. Additionally, as time progresses and new tools become available, product comparisons are drawn up. Hardion et al. \cite{Hardion2013} for example present a high level comparison of various configuration management systems on the market from the standpoint of a large research facility. However, none of these reports mention a migration strategy between systems.

This research primarily focuses on a migration trajectory in which a transition is made between Puppet and Ansible and vice versa, as corporate policy or functional requirements may dictate the final configuration management tool. Little official research related to this specific topic is currently available. Many parties only present a motivation for moving away from a given tool to a (usually) newer and more advanced tool but don't publicly present their strategy. Dawson et al. \cite{dawson_hall_hecht_2014} for example present a motivation from moving away from Puppet to Ansible and argue that the latter is superior to other tools with regards to deploying containers. Additionally, they identify Puppet as one of the most influential configuration management systems of current time. Ansible as a tool is identified as a common migration target due to its easier to grasp language constructs and its simpler mode of operation. More detail on the advantages and languages of each tool can be found in section \ref{sec:background}. 
\\\\
On the topic of migration strategies, Zunker \cite{zunker_2014} goes into more depth in a blog post by presenting a starting point for performing a gradual migration between Puppet and Ansible. He describes a method for executing Puppet modules through an Ansible playbook. By using this strategy Ansible temporarily functions as a remote executor for Puppet during a period in which all Puppet modules are being converted to Ansible playbooks. A major downside of this strategy however is that Ansible can't perform correct error handling for Puppet modules being run through Ansible, therefore making this a risky migration strategy. Hasib \cite{hasib_2015} writes about a different strategy used within the company NewsCred for migrating from Puppet to Ansible by leveraging 'Fabric', another push-based configuration management system. In their environment Puppet was implemented in an agentless form, which means that there is no central Puppet Master overlooking the IT environment. Fabric python scripts were executed to SSH into a machine and call specific Puppet commands in parallel on each node in their inventory. Migrating their infrastructure over to Ansible was relatively easy as it only required rewriting Puppet modules to Ansible playbooks. After performing their migration Fabric was still being used to execute Ansible commands. Naturally, not every company has the luxury of having a hierarhical setup of configuration management systems in place. Therefore this migration method is limited to a select few companies. Additionally, migrating away from the wrapper CMS, Fabric, would present similar issues as a normal migration between systems. Arguably migrating away might even be more difficult due to the hierarchy involved. 

As of late, new concepts for configuration management tools have been surfacing. A prime example being \texttt{mgmt} by Shubin \cite{shubin2016}. \texttt{Mgmt} is a self proclaimed 'next generation' configuration management prototype. Up until now, most configuration management tools have been employing a push or pull-based client-server model. \texttt{Mgmt} on the other hand uses a distributed architecture which theoretically allows for better scalability and resilience. Due to its age and prototype status, no official research is available regarding \texttt{mgmt} at this point in time. However, preliminary tests have been publicized by Shubin in which the speed of convergence and technical proof of concepts are presented. The measurements presented by Shubin will form a starting point for the tests performed in this report regarding distributed configuration management systems. 
\\\\
Lastly, converting desired state configurations from one CMS to another poses another problem. Commonly each CMS utilises a different DSL to define a state which makes a migration manual labor. The same problem exists for Puppet modules (Ruby-based) and Ansible playbooks (Python-based). Nigmatullin and Van Dijk \cite{marat_2016} are currently working on automating the creation of these playbooks and modules from a static server configuration. Additionally, Frank \cite{frank_2016} is making similar efforts for converting Puppet to \texttt{mgmt}. More information on this topic can be found in section \ref{subsec:futurework}. 