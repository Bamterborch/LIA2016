\section{Related work}\label{sec:relatedwork}
Configuration management tools are a frequently researched topic by Enterprises, as their presence in an IT landscape  may lead to improvements in overall workflow. As these tools are mainly aimed at corporate infrastructures, the available literature commonly focuses on specific use cases from an Enterprise standpoint. Frequently these reports present use cases related to security aspects such as infrastructure hardening \cite{dotson2014security} or the repeatability of deployments \cite{ruiz2015reconstructable}. Additionally, as time progresses and new tools become available, product comparisons are drawn up. Hardion et al. \cite{Hardion2013} present a high level comparison of various configuration management tools on the market from the standpoint of a large research facility. Regarding the actual configuration of such tools, Collard et al. \cite{Collard2015} describe a method for verifying the configuration of Puppet. However, this paper exclusively focuses on environments with a single configuration management system. 

This research focuses on a migation trajectory in which a transition is made between Puppet and Ansible and vice versa, as corporate policy or functional requirements may dictate the final configuration management tool. Little official research related to this specific topic is currently available. Many parties only present a motivation for moving away from a given tool to a (usually) newer and more advanced tool. Dawson et al. \cite{dawson_hall_hecht_2014} presents such a motivation and argues that Ansible is superior to other tools with regards to deploying containers. Additionally, he identifies Puppet as one of the most influential configuration management tools of current time. Ansible as a tool is identified as a common migration target. 

On the topic of migation strategies, Zunker \cite{zunker_2014} goes into more depth in a blog post and presents a starting point for performing a gradual migration between Puppet and Ansible. He describes a conceptual method for piping output from Puppet to Ansible and talks about interpreting the output from each tool.

Moreover, as of late, new concepts for configuration management tools have been surfacing. \texttt{mgmt} by Shubin \cite{shubin2016} is a self proclaimed 'next generation' configuration management prototype.  Up until now, most configuration management tools have been employing a push or pull-based client-server model. \texttt{mgmt} uses a distributed architecture which allows for parallelization and a distributed topology. Due to the way \texttt{mgmt} is built, the tool may be suitable for a gradual migration trajectory.


%Talk about Zunker executing Puppet from within Ansible --> Not desirable as the end migration is not a single system. (Refer to discussion, as there ARE use cases for mixed environment)
https://blog.codecentric.de/en/2014/12/migrate-puppet-ansible/
