\section{Introduction}\label{sec:introduction}
In order to keep up with the growing demand for a broad range of IT services, many companies are looking into the scalability of the services they provide. As a result of a cost evaluation, a common strategy of scaling an application at this point in time is to scale out horizontally \cite{jaapvginkel}. This inherently means that the total amount of nodes in an IT environment increases rapidly. In order to achieve and more importantly, maintain such an environment a large number of Enterprises are moving towards a 'DevOps' culture. This movement is built around the idea of operations engineers and software developers tightly working together in order to provide a programmable and dynamic infrastructure which focuses on automated deployments \cite{loukides_2012}. Gartner believes that by now this strategy has evolved from a niche market to a mainstream strategy \cite{gartner_2015}. The automation aspect of this shift in culture is largely assisted by configuration management systems (CMS) such as Puppet \cite{whatispuppet}, Ansible \cite{whatisansible}, Chef \cite{whatischef} and tools alike. These tools allow one to configure a large, heterogeneous IT landscape by leveraging Domain Specific language (DSL) constructs, effectively defining your infrastructure in code. As such, the repeatability and speed of service deployments can be greatly enhanced whilst simultaneously allowing a company to move towards a Service Oriented Architecture (SOA).



The concept of configuration management systems predates the relatively new 'DevOps' terms. However, these tools are more relevant than ever. With an ever increasing IT landscape, many companies are looking into, or have already implemented an established configuration management tool such as Puppet or Chef. Additionally, as time progresses and new requirements are being identified, tools like Salt and Ansible are introduced to fill the void. However, moving from a given CMS to another in case of a corporate merge for example, is a complicated process. Commonly in such scenarios, one of the tools is identified to be the ultimate tool managing the entire environment for uniformity sake. This project focuses on the migration phase between configuration management systems. Naturally, switching a large environment between management tools requires a clear strategy and preferably the migration is performed without noticeable impact from a user perspective. Downtime of a service is out of the question. The implications of switching a node in an automated fashion back and forth between management tools requires further investigation. 

Therefore, the aim of this research project is to answer the following main research question:

\begin{quote}
\textit{How can a node be automatically migrated between two configuration management systems?}
\end{quote}

\noindent
Several sub-questions have been posed to support this rather broad main question:

\begin{itemize}
	\setlength\itemsep{1pt}
    \item \textit{Which steps are involved in transitioning a node between configuration management systems with a varying architecture?}
    \item \textit{Is a migration towards a distributed configuration management system feasible within the foreseeable future?}
\end{itemize}

\noindent
An automatic migration in the sense of this report refers to a seamless migration between management systems in which a node is migrated, in production, without any perceivable consequences from an end user perspective. This means that the service on the node needs to stay available in the same way as it is before the migration, not considering other redundancy mechanisms like load balancers of hot-standby devices. The scope of this research is limited by exclusively examining the configuration management tools Puppet and Ansible. Puppet is an established tool, commonly deployed in production environments \cite{tecosystems_2013}. Ansible on the other hand is a more recent tool and is gaining traction in corporate environments at a rapid rate.

Lastly, this report examines the innovations which \texttt{mgmt} brings to the table. \texttt{Mgmt} is a new prototype configuration management system which introduces the concepts of parallelization, distributed architectures and event driven convergence to configuration management. An evaluation is made of its applicability in the current state of the project and possible migration strategies towards this system are discussed.
