\section{Introduction}\label{sec:introduction}
For this puprose, a large amount of Enterpises are currently moving towards a DevOps culture. This movement focuses on the idea of operations engineers and software developers tightly working together in order to provide a programmable and dynamic infrastructure which focuses on automated deployments. Gartner believes that this strategy will evolve from a niche market to a mainstream strategy. 

Configuration mangement systems like CFengine and [] predate the term 'DevOps', continued to grow. Now, many companies are at a crossway. Implement established tools such as Puppet or Chef, or go for the new projects like Ansible and Salt? Established enterprises of scale commonly already have configuration management tools. A corporate merge or an acquisition may mean that one of the tools becomes final. Which strategies are currently available and what do we think? We are interested in the implications of switching a node in an automatic fashion between management tools. 

% Are there scenarios in which both Puppet and Asible can be deployed in tandem?

For this puprose, a large amount of Enterpises are currently moving towards a DevOps culture. This movement focuses on the idea of operations engineers and software developers tightly working together in order to provide a programmable and dynamic infrastructure which focuses on automated deployments. Gartner believes that this strategy will evolve from a niche market to a mainstream strategy. 

This shift in culture is largely assisted by configuration management systems such as Puppet \cite{whatispuppet}, Ansible \cite{whatisansible} and Chef \cite{whatischef}. 

Therefore, the aim of this research project is to answer the following main research question:

\begin{quote}
\textit{How can an operating system be automatically migrated between two configuration management systems?}
\end{quote}

\noindent
The scope of this research is limited by exclusively examining the configuration management tools Puppet and Ansible. Puppet currently proves to have the most commercial backing and is most likely to be deployed in production environments [SOURCE]. Ansible is a more recent tool and it is gaining traction in corporate environments at a rapid rate. It is being developed by an active community and is backed by Red Hat. Additionally, since current configuration management systems are effectively limited in scalability due to their architectural model, we discuss the distributed architecture of a \texttt{mgmt}, which at this point in time is a prototype configuration management system.
