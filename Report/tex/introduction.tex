\section{Introduction}\label{sec:introduction}
In order to keep up with the growing demand for a broad range of IT services, many companies are looking into the scalability of the services they provide. As a result of a cost evaluation, a common strategy of scaling an application at this point in time is to scale out horizontally. This inherently means that the amount of nodes in an IT environment increases rapidly. In order to achieve and -more importantly- maintain such an environment, a large number of Enterprises are moving towards a 'DevOps' culture. This movement focuses on the idea of operations engineers and software developers tightly working together in order to provide a programmable and dynamic infrastructure which focuses on automated deployments \cite{loukides_2012}. Gartner believes that by now this strategy has evolved from a niche market to a mainstream strategy \cite{gartner_2015}. The automation aspect of this shift in culture is largely assisted by configuration management systems (CMS) such as Puppet \cite{whatispuppet}, Ansible \cite{whatisansible}, Chef \cite{whatischef} and tools alike. These tools allow one to configure a large, heterogeneous IT landscape by leveraging Domain Specific language (DSL)  constructs, effectively defining your infrastructure in code. As such, the repeatability and speed of service deployments can be greatly enhanced whilst simultaneously allowing a company to move towards a Service Oriented Architecture (SOA).

The concept of configuration management systems predates the relatively new 'DevOps' terms. However, these tools are more relevant than ever. With an ever increasing IT landscape, many companies are looking into, or already have implemented an established configuration management tool such as Puppet or Chef. Additionally, as time progresses and new requirements are being identified, tools like Salt and Ansible are introduced to fill the void. However, as newIT requirements arise, the need to move from a given CMS to a new one arises. Problems occur when a corporate merger occurs, due to an acquisition for example. In such scenario, one of the tools can be chosen to manage the entire environment for uniformity sake. We are especially interested in the latter scenario. Naturally, switching a large environment between management tools requires a clear strategy. The implications of switching a node in an automated fashion back and forth between management tools requires further investigation. 

Therefore, the aim of this research project is to answer the following main research question:

\begin{quote}
\textit{How can an operating system be automatically migrated between two configuration management systems?} %Nog niet helemaal over eens :)
\end{quote}

\noindent
The scope of this research is limited by exclusively examining the configuration management tools Puppet and Ansible. Puppet currently proves to have the most commercial backing and is most likely to be deployed in production environments [SOURCE]. Ansible on the other hand is a more recent tool and is gaining traction in corporate environments at a rapid rate. It is being developed by an active community and is backed by Red Hat. Additionally, since current configuration management systems are effectively limited in scalability due to their architectural model, we briefly discuss the distributed architecture of \texttt{mgmt}, which at this point in time is a prototype configuration management system.

Migrating between systems can be done in different strategies, when using Configuration management system (CMS) A on all of the servers. New servers can be managed by CMS B. This will slowly migrate between the systems. The lifetime of a server dictates the time to migrate, nowadays only the low level systems like NTP, DNS and authentication server should be bare metal servers. All other services depending on these could be virtual. Therefore the lifetime of these virtual servers is often less than the lifetime of bare metal machines. [SOURCES] Since we use Virtual machines the migration process could be finished a lot faster. In a cluster of 10 webservers managed by CMS A offering the same website, a new webserver could be created managed by CMS B. When tested and approved it should be added to the cluster. An old server should be removed from the cluster and decommissioned. In Large scale environments this will be an extensive job.
To keep server management easy and effective. One system should be used and the migration should not stretch multiple years. A Big bang scenario is a proper alternative. Migrating all the webservers from CMS A to CMS B by a set of automated steps without installing new servers. Therefore not using extra resources and migrating without downtime. This big bang scenario is explained in this report as the focus of this report lays on Large Installation Adminstration.

\begin{quote}
According to Jaap van Ginkel, "More people scale out services by adding virtual servers to the service." 
\end{quote}
     
% Talk about Big bang migraton vs phased migration
http://www.data-migrations.com/strategy.html

