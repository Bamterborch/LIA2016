\section{Introduction}\label{sec:introduction}
Providing standardized services 

In order to achieve this, a large number of Enterpises are currently moving towards a DevOps culture. This movement focuses on the idea of operations engineers and software developers tightly working together in order to provide a programmable and dynamic infrastructure which focuses on automated deployments. Gartner believes that by now this strategy has evolved from a niche market to a mainstream strategy. The automation aspect of this shift in culture is largely assisted by configuration management systems such as Puppet \cite{whatispuppet}, Ansible \cite{whatisansible} and Chef \cite{whatischef}. These tools allow you to configure a large, heterogeneous IT landscape by leveraging Domain Specific language (DSL)  constructs, effectively defining your infrastructure in code. As such, the repeatability and speed of service deployments can greatly be enhanced. 

The concept of configuration management systems predates the relatively new 'DevOps' terms. However, they are more relevant than ever.  Many companies are finding themselves at a crossway. One can choose to implement an established configuration mangement tool such as Puppet or Chef, or opt for newer initiatives like Salt or Ansible? Additonally, established enterprises of scale commonly already have configuration management tools in place. In the case of a corporate merger -due to an acquisition for example- one of the tools has to be chosen to manage the environment. We are interested in the implications of switching a node in an automatic fashion between management tools. 

Therefore, the aim of this research project is to answer the following main research question:

\begin{quote}
\textit{How can an operating system be automatically migrated between two configuration management systems?}
\end{quote}

\noindent
The scope of this research is limited by exclusively examining the configuration management tools Puppet and Ansible. Puppet currently proves to have the most commercial backing and is most likely to be deployed in production environments [SOURCE]. Ansible on the other hand is a more recent tool and is gaining traction in corporate environments at a rapid rate. It is being developed by an active community and is backed by Red Hat. Additionally, since current configuration management systems are effectively limited in scalability due to their architectural model, we discuss the distributed architecture of \texttt{mgmt}, which at this point in time is a prototype configuration management system.
