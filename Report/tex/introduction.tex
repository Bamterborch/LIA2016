\section{Introduction}\label{sec:introduction}
In order to keep up with the growing demand for IT services across the board, many companies are looking into the scalability of the services they provide. A common strategy of scaling an application is to scale out horizontally. This inherently means that the amount of nodes in an IT environment increases rapidly. In order to achieve and maintain such an environment, a large number of Enterprises are moving towards a 'DevOps' culture. This movement focuses on the idea of operations engineers and software developers tightly working together in order to provide a programmable and dynamic infrastructure which focuses on automated deployments \cite{loukides_2012}. Gartner believes that by now this strategy has evolved from a niche market to a mainstream strategy \cite{gartner_2015}. The automation aspect of this shift in culture is largely assisted by configuration management systems such as Puppet \cite{whatispuppet}, Ansible \cite{whatisansible}, Chef \cite{whatischef} and alike. These tools allow one to configure a large, heterogeneous IT landscape by leveraging Domain Specific language (DSL)  constructs, effectively defining your infrastructure in code. As such, the repeatability and speed of service deployments can greatly be enhanced whilst simultaneously allowing a company to move towards a Service Oriented Architecture (SOA).

The concept of configuration management systems predates the relatively new 'DevOps' terms. However, these tools are more relevant than ever. With an ever increasing IT landscape, many companies are looking into, or already have implemented an established configuration management tool such as Puppet or Chef. Additionally, as time progresses and new requirements are being identified, tools like Salt and Ansible are introduced to fill the void. Problems occur when a corporate merger occurs, due to an acquisition for example. In such scenario, one of the tools can be chosen to manage the entire environment for uniformity sake. We are especially interested in the latter scenario. Naturally, switching a large environment between management tools requires a clear strategy. The implications of switching a node in an automated fashion back and forth between management tools requires further investigation. 

Therefore, the aim of this research project is to answer the following main research question:

\begin{quote}
\textit{How can an operating system be automatically migrated between two configuration management systems?}
\end{quote}

\noindent
The scope of this research is limited by exclusively examining the configuration management tools Puppet and Ansible. Puppet currently proves to have the most commercial backing and is most likely to be deployed in production environments [SOURCE]. Ansible on the other hand is a more recent tool and is gaining traction in corporate environments at a rapid rate. It is being developed by an active community and is backed by Red Hat. Additionally, since current configuration management systems are effectively limited in scalability due to their architectural model, we briefly discuss the distributed architecture of \texttt{mgmt}, which at this point in time is a prototype configuration management system.
