-\section{Discussion}\label{sec:discussion}
-During this project we have performed an in place migration from Puppet to Ansible and vice versa with a single CMS managing the whole infrastructure as a result. Arguably however, there are use cases in which both systems can coexist with varying purposes in mind. Villamil \cite{} for example argues that Ansible is more suitable for server deployments and installation whereas Puppet excels at defining an end state for a system. From a technical standpoint, such a 'migration trajectory' is significantly easier as it omits the automated removal step from the migration strategy as portrayed in section \ref{sec:methodology}. However, the migration strategy as presented in this report is not a 'one size fits all' soluion. One could argue for example that when a large amount of resources are available, an environment could be migrated by spinning up copies of existing servers with the new configuration management tool in place. Especially when a loadbalancer is in place, these servers could be deployed in parralel with existing servers. The older machines could then be gradually decomissioned. However, this would only work when an excess amount of resources are available and when the total amount of servers to be migrated is manageable, as otherwise a gradual migration strategy would require a disproportional amount of time. Additionally with the move to DevOps, some companies might employ an immutability concept when deploying servers meaning that any change to a server configuration would require the base image file to be redefined. Subsequently, the server would get trashed and redeployed with the new configuration. 
-
-As a closing remark we want stress that a migration between configuration management systems is a delicate task in which specific configuration changes have to be made in a specific order. Addtionally, these changes depend heavily on the architectural model employed by the tool in place. Therefore it is not advisable to envision the migration steps in this report as a generic migration model.  
-
-http://t37.net/should-i-use-ansible-or-puppet-short-answer-both.html
