\section{Methodology}\label{sec:methodology}

% insert image for situation 1. From puppet to Ansible 

In case of a corporate merge where both companies are using a server management system. Company A uses Puppet and company B uses Ansible. When the decision is made to use Ansible as the new configuration management system all clients under the Puppet-master need to be added to the Ansible environment. But before this can be done, some prerequisites need to be matched in order not to loose functionality.
As explained in chapter \ref{subsec:background} Puppet and Ansible are different in the way they send commands to the clients. Where Puppet is a pull based management system, Ansible is a push based management system. In short: Puppet clients check in to the Puppet-master to see if there are any new jobs to execute and to see if the configuration is still in the correct state. Ansible connects to the clients over a SSH session to deliver new jobs to execute. In figure \ref{fig:situation1} client group "webservers" is under control of the Puppet-master. The Ansible server needs to be able to connect to the webservers in this client group in a network technical way over TCP port 22. All packages managed by puppet need to be reproduces in Ansible. The reason for this is that no functionality may be removed as a result of the migration to the new system. 

So first, the clients need to be added to Ansible as a group, this should be done by puppet. To make this work. The puppet client needs to be temporary installed on the Ansible server in order to send jobs to this server. The Ansible hosts file need to be updated with the servers from the puppet webservers group. Next to this job, the puppet clients need to be told to remove the puppet client from the system and allow ssh connections from the Ansible server. The new playbook (as the set of commands is called within Ansible) is made and tested before changing the management system. When the puppet webservers are added to the Ansible system and the puppet client is removed from the clients the created playbook should be run automatically. The puppet client on the Ansible machine needs to run a job that was created by the puppet-master that tells to start the Ansible playbook that will take over the server. After all servers are merged from puppet to Ansible, the last puppet job can be run. The puppet-client needs to be removed from the Ansible server. This should be done in the same way as this is done on the client machines. To play the Ansible playbook a simple command should be executed. 

% insert refs to seperate parts. especially he parts where ansible and puppet need to be communicating together. 

% insert image for situation 2. From Ansible to puppet

If the decision of the configuration management system goes into the favor of Puppet, all machines managed by Ansible need to receive a Puppet client. This should be done by an Ansible playbook. There are multiple ways to install the Puppet client onto the client group "webservers". A playbook can be created with the command to install the client using the bash installer provided by the Puppet-master. This way is used during this small project, the playbook is shown at figure \ref{fig:puppetplaybook} An other way is to use the Puppet module within Ansible \cite{ansiblepuppet}  
